\section{PETSc to Other Packages}

PETSc uses run-time binding of data structures and solver algorithms which means
that in most causes you do not need to change your simulation code to switch
between solvers in hypre, SuperLU, and Trilinos. You use the PETSc options
database (for example command line arguments) to select your solvers; it is also
possible to hardware particular solvers into the source code but we do not
recommend this approach. The solvers in the other xSDK packages are interfaced
to PETSc in the PETSc abstract {PC} preconditioners interface\footnote{This same
interface is used for direct solvers}. To select a particular preconditioner one uses

\begin{verbatim}
-pc_type typename [-pc_* other potential options related to the solver]
\end{verbatim}


\subsection{PETSc utilizing SuperLU}

PETSc can use both SuperLU and SuperLU\_Dist solvers. Since the SuperLU\_Dist
solvers are parallel and in general are faster and require less memory than
SuperLU we will only discuss the details of utilizing SuperLU\_Dist. It is
utilized with


\begin{verbatim}
-pc_type lu -pc_factor_mat_solver_package superlu_dist [-mat_superlu_dist_* 
SuperLU_Dist options]
\end{verbatim}
All the options will be printed if you run with {\tt -pc\_type lu
-pc\_factor\_mat\_solver\_package superlu\_dist -help | grep superlu\_dist}. We
also display many of the options in Table \ref{table:superlu_parameters}.

\begin{table}
\center
\begin{tabular}{p{2in} p{.75in} p{.5in} p{2in}}
  \hline
  Option  & Arguments & Default & Description  \\
  \hline
  -mat\_superlu\_dist\_statprint  & Yes/No & No & Prints statistics on factorization\\
  {\color{red}TODO: Someone needs to with -help and add the rest.} \\
  \hline
\end{tabular}
\caption{PETSc SuperLU\_Dist Options}
\label{table:superlu_parameters}
\end{table}

\subsection{PETSc utilizing hypre}

PETSc utilizes several of the solvers/preconditioners in hypre via the command

\begin{verbatim}
-pc_type hypre -pc_hypre_type [boomeramg (default),pilut,parasails,ams,ads]
\end{verbatim}

Since BoomerAMG is the crown jewel of hypre we recommend its use whenever
possible and do not discuss here the use of the other solvers. Running with {\tt
-pc\_type hypre -help | grep hypre} for example will show the options for the
{\tt pilut} preconditioner.  Many of the options are described in Table
\ref{table:hypre_parameters}.

\begin{table}
\center
\begin{tabular}{p{2in} p{.75in} p{.5in} p{2in}}
  \hline
  Option  & Arguments & Default & Description  \\
  \hline
  -pc\_hypre\_boomeramg\_max\_levels  &  l & Problem dependent & Sets maximum
  number of levels\\
  {\color{red}TODO: Someone needs to with -help and add the rest.} \\
  \hline
\end{tabular}
\caption{PETSc hypre BoomerAMG Options}
\label{table:hypre_parameters}
\end{table}

\subsection{PETSc utilizing Trilinos}

PETSc can utilize the ML solver in Trilinos.

\begin{verbatim}
-pc_type ml [-pc_ml_* options for ML]
\end{verbatim}

 Running with {\tt -pc\_type ml -help | grep pc\_ml}  will show the available
 options, many of which are detailed in Table \ref{table:trilinos_parameters}.

\begin{table}
\center
\begin{tabular}{p{2in} p{.75in} p{.5in} p{2in}}
  \hline
  Option  & Arguments & Default & Description  \\
  \hline
  -pc\_ml\_maxNlevels  &  l & 10 & Sets maximum number of levels\\
  {\color{red}TODO: Someone needs to with -help and add the rest.} \\
  \hline
\end{tabular}
\caption{PETSc ML Options}
\label{table:trilinos_parameters}
\end{table}

{\color{red}TODO: Is this the only interface PETSc has to Trilinos?}
